\documentclass[40pt,a4paper]{article}

\usepackage{graphicx, amsmath, amsfonts, listings}
\usepackage[T2A]{fontenc}
\usepackage[utf8]{inputenc}
\usepackage[russian,english]{babel}
\usepackage{geometry, tocloft}

\geometry{a4paper,left=15mm,top=15mm}

\graphicspath{{./images/}}

% Переименование оглавления
\addto\captionsenglish{\renewcommand{\contentsname}{Содержание}}

% Настройки межстрочных отступов в оглавлении
\setlength\cftparskip{1.2pt}
\setlength\cftbeforesecskip{1.3pt}
\setlength\cftaftertoctitleskip{2pt}

% Добавление точки после номера раздела
\makeatletter
\renewcommand{\@seccntformat}[1]{\csname the#1\endcsname.\quad}
\makeatother

% Добавление точек в оглавление
\let\savenumberline\numberline
\def\numberline#1{\savenumberline{#1.}}


\begin{document}
	
	\begin{titlepage}
		\centering
		
		{\LARGE\bfseries Министерство образования и науки Российской Федерации}
		
		\vspace{0.5cm}
		
		{\LARGE\bfseries Санкт-Петербургский политехнический университет Петра Великого
		}
		
		\vspace{1.5cm}
		
		{\Large Отчет по лабораторной работе №1 по дисциплине}
		{\Large “Интервальный анализ”}
		
		\vspace{1cm}
		
		{\Large \textit{Вырожденность интервальной матрицы}}
		
		\vspace{16cm}
		
		\begin{center}
			\begin{tabular}{ r l c }
				Выполнил                       \\ 
				студент гр. 5030102/20201 & Н.А.Морщинин \\  
				преподаватель             & А.Н.Баженов
			\end{tabular}
		\end{center}
		\vfill
		
	\end{titlepage}
	
	\thispagestyle{empty} 
	
	\newpage
	
	\tableofcontents
	
	\newpage
	\section{Введение}
	\subsection{Постановка задачи}
	Дана прямоугольная матрица А с параметром $\delta$, элементами которой являются интервалы $[a_{ij},b_{ij}]$. 
	Необходимо найти минимальный параметр $\delta>0$, при котором определитель подматрицы $[d_a, d_b]$ будет содержать 0:
	
	\begin{math}
		\begin{vmatrix}
			A
		\end{vmatrix}
		= 
		\begin{vmatrix}
			A(\delta)
		\end{vmatrix}
		= [d_a, d_b]:
		0 \in [d_a, d_b]
	\end{math}
	
	\subsection{Условия}
	\begin{math}
		A_1 = 
		\begin{pmatrix}
			[0.95-\delta, 0.95+\delta] & [1-\delta, 1+\delta] \\
			[1.05-\delta, 1.05+\delta] & [1-\delta, 1+\delta] \\
			[1.10-\delta, 1.10+\delta] & [1-\delta, 1+\delta] \\
		\end{pmatrix}
		,
		A_2 = 
		\begin{pmatrix}
			[0.95-\delta, 0.95+\delta] & [1,1] \\
			[1.05-\delta, 1.05+\delta] & [1,1] \\
			[1.10-\delta, 1.10+\delta] & [1,1] \\
		\end{pmatrix}
	\end{math}
	
	\section{Теория}
	\subsection{Определитель квадратной матрицы}
	\paragraph{Для квадратной матрицы $A^{2,2}$ может быть вычислен следующим образом:}
	
	\begin{math}
		\begin{vmatrix}
			a & b \\
			c & d \\
		\end{vmatrix}
		= ad-bc
	\end{math}
	
	
	\paragraph{Для квадратной матрицы большей размерности существует общая формула вычисления определителя: }
	\paragraph{Для любой строки $i$:}
	
	\begin{math}
		\det(A) = \sum_{j=1}^n (-1)^{i+j} a_{ij} \cdot M_{ij}
	\end{math}
	
	\paragraph{Для любого столбца $j$:}
	
	\begin{math}
		\det(A) = \sum_{i=1}^n (-1)^{i+j} a_{ij} \cdot M_{ij}
	\end{math}
	
	\paragraph{где $M_{ij}$ — дополнительный минор элемента $a_{ij}$.}
	
	\subsection{Определитель прямоугольной матрицы}
	
	Для вычисления определителя прямоугольной матрицы в данной задаче нужно перебрать все квадратные подматрицы матрицы $A$, порядка, не меньшего чем $min(n,m)$, и проверить условие на содержание нуля для всех определителей.
	
	\subsection{Операции с интервалами}
	
	Для интервалов вида $[a,b]$, $a<b$ определены следующие операции:
	\begin{itemize}
		\item \textbf{Сложение ``+''}: $[a,b] + [c,d] = [a + c, b + d]$
		
		\item \textbf{Вычитание ``-''}: $[a,b] - [c,d] = [a - d, b - c]$
		
		\item \textbf{Умножение ``*''}: 
		$[a,b] \cdot [c,d] = [\min\{ac, ad, bc, bd\}, \max\{ac, ad, bc, bd\}]$
		
		\item \textbf{Деление ``/''}: 
		$[a,b] / [c,d] = [a,b] \cdot [1/d, 1/c]$, при условии $0 \notin [c,d]$
	\end{itemize}
	
	\newpage
	
	\section{Реализация}
	
	\subsection{Алгоритм}
	
	Пусть есть одномерная функция f, принимающая два значения \{0, 1\}:
	$
	f(\delta) = (0 \in 
	\begin{vmatrix}
		A(\delta)
	\end{vmatrix})
	$
	
	Тогда для решения задачи нужно найти такое минимальное значение, при котором f имеет скачок и меняет свое значение с 0 на 1.
	
	Решение реализуем с помощью метода пробных точек.
	
	\subsection{Псевдокод алгоритма}
	
	Вход:
	\begin{enumerate}
		\item A, B (границы отрезка, в котором ищем $\delta$)
		\item Точность поиска $\epsilon$
		\item f - функция для вычисления
	\end{enumerate}
	Выход: 
	Значение $\delta = min _{\delta \in [A,B]} (f(\delta))=1$
	
	\begin{enumerate}
		\item Инициализация $n$: $n = \frac{B-A}{\epsilon}$ - количество точек для разбиения отрезка.
		\item Инициализация $x_i, y_i$ ($x_i$ строятся по равномерной сетке, $y_i = f(x_i), i  = \overline{1,n}$).
		\item Проверка условия: $|b - a| > 2 \epsilon$, если условие не выполняется, то выход из итераций вывод ответа.
		\item Итерация по $x_i$:
		\begin{enumerate}
			\item Если для некоторого $i\neq 1: f(x_{i-1})=0 \ \&\ f(x_i)=1$, то нашли точку скачка, уточняем отрезок: $A=x_{i-1}, B=x_i$, переходим к шагу 1.
			\item Иначе продолжаем итерации.
		\end{enumerate}
	\end{enumerate}
	
	\newpage
	
	\section{Выводы}
	
	\subsection{Результат}
	
	Для $A_1: \delta = 0.013$ \newline
	Для $A_2: \delta = 0.075$
	
	\subsection{Интерпретация результата}
	
	По значениям $\delta$ видно, как быстро матрица с такими значениями начинает превращаться в вырожденную.
	
\end{document}


