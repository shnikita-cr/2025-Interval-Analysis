\documentclass[40pt,a4paper]{article}

\usepackage{graphicx, amsmath, amsfonts, listings}
\usepackage[T2A]{fontenc}
\usepackage[utf8]{inputenc}
\usepackage[russian,english]{babel}
\usepackage{geometry, tocloft}
\usepackage{hyperref}

\geometry{a4paper,left=15mm,top=15mm}

\graphicspath{{./images/}}

% Переименование оглавления
\addto\captionsenglish{\renewcommand{\contentsname}{Содержание}}

% Настройки межстрочных отступов в оглавлении
\setlength\cftparskip{1.2pt}
\setlength\cftbeforesecskip{1.3pt}
\setlength\cftaftertoctitleskip{2pt}

% Добавление точки после номера раздела
\makeatletter
\renewcommand{\@seccntformat}[1]{\csname the#1\endcsname.\quad}
\makeatother

% Добавление точек в оглавление
\let\savenumberline\numberline
\def\numberline#1{\savenumberline{#1.}}

\newcommand{\ran}{\mathrm{ran}}
\newcommand{\rad}{\mathrm{rad}}
\newcommand{\dist}{\mathrm{dist}}

\begin{document}

\begin{titlepage}
\centering

{\LARGE\bfseries Министерство образования и науки Российской Федерации}

\vspace{0.5cm}

{\LARGE\bfseries Санкт-Петербургский политехнический университет Петра Великого
}

\vspace{1.5cm}

{\Large Отчет по лабораторной работе №4 по дисциплине}
{\Large “Интервальный анализ”}

\vspace{1cm}

{\Large \textit{Вычисление интервальных описательных статистик}}

\vspace{16cm}

\begin{center}
\begin{tabular}{ r l c }
 Выполнил                       \\ 
 студент гр. 5030102/20201 & Н.А.Морщинин \\  
 преподаватель             & А.Н.Баженов
\end{tabular}
\end{center}
\vfill

\end{titlepage}

\thispagestyle{empty} 

\newpage

\tableofcontents

\newpage
\section{Введение}
\subsection{Постановка задачи}


Даны два входных файла данных диагностики томпсоновского рассеяния.


Связь кодов данных и Вольт для преобразования единиц измерения выражается следующим образом: 


$V = Code / 16384 - 0.5$


По данным из входных файлов необходимо реализовать следующее:

\begin{enumerate}
    \item Пусть $X$ и $Y$ - интервальные выборки вида $\bold{X} = \{x_i\}$, $\bold{Y} = \{y_i\}$
    
    
    Извлечь $X$ и $Y$ из данных входных файлов, задав $rad x = rad y = \frac{1}{2^N}, N = 14$
    
    \item Пусть зависимость $Y$ и $X$ задается следующими выражениями:
    
    $a + \bold{X} = \bold{Y}$
    
    
    $t * \bold{X} = \bold{Y}$
    
    Вычислить точечные и интервальные оценки констант $a$, $t$, в уравнениях с помощью некоторого функционала $F$, задавшись уровнем точности $\epsilon$:
    
    $\hat{s} = argmax\ F(s, \bold{X}, \bold{Y})$, где $s \in \{a,t\}$
    
    
    Для функционала $F$ рассмотреть следующие случаи:
    
    \begin{enumerate}
    	\item $F(s, \bold{X}, \bold{Y}) = Ji(s,\bold{X}, \bold{Y})$
    	\item $F(s, \bold{X}, \bold{Y}) = Ji(s,mode\bold{X}, mode\bold{Y})$
    	\item $F(s, \bold{X}, \bold{Y}) = Ji(s,med_K\bold{X}, med_K\bold{Y})$
    	\item $F(s, \bold{X}, \bold{Y}) = Ji(s,med_P\bold{X}, med_P\bold{Y})$
    \end{enumerate}
    
    где Ji - коэффициент Жаккара, mode - интервальная мода, $med_K,\ med_P$ - интервальные медианы Крейновича и Пролубникова.
    
    \item Для каждого пункта a-d предоставить графики $F(s)$, отметить $s_{max}$.
    \item Сравнить полученные результаты.
\end{enumerate}

\subsection{Условия задачи}

\subsubsection{Исходные данные}

Данные представлены в файлах: 

\begin{itemize}
	\item \textit{-0.205\_lvl\_side\_a\_fast\_data.bin}
	\item \textit{0.227\_lvl\_side\_a\_fast\_data.bin}
\end{itemize}

\subsubsection{Предобработка данных}

Для извлечения данных читался каждый файл и записывались значения каждого канала с точки каждого кадра в структуру данных, затем преобразованы по формуле для получения значений в Вольтах.



Вторым шагом данные каждого файла "выпрямлялись": значения с каналов точек кадров $s$ помещались в единый массив данных $v$, таким образом, что 


$v[i]=s[F][P][C]$. 

Таким образом были получены точечные вектора $Xd$, $Yd$.

Вектор брусы были получены из точечных добавлением неопределенности радиуса $r$:

$V[i] = [v[i]-r, v[i]+r]$

Таким образом были получены вектор брусы $X$, $Y$.

\newpage




\section{Теория}

\subsection{Коэффициент Жаккара для интервалов}

Для двух интервалов $A=[\underline{a},\overline{a}]$, $B=[\underline{b},\overline{b}]$:
\[
|A|=\max(0,\overline{a}-\underline{a}),\quad
|B|=\max(0,\overline{b}-\underline{b}),
\]
\[
|A\cap B|=\max\!\bigl(0,\min(\overline{a},\overline{b})-\max(\underline{a},\underline{b})\bigr),
\]
\[
J(A,B)=\frac{|A\cap B|}{|A|+|B|-|A\cap B|}.
\]

Для двух интервальных векторов одинаковой размерности:
\[
J(\mathbf{A},\mathbf{B})=\frac{\sum_{i=1}^{n} |A_i\cap B_i|}{\sum_{i=1}^{n} \bigl(|A_i|+|B_i|-|A_i\cap B_i|\bigr)}.
\]

\subsection{B1}

\[
F(a,\mathbf{X},\mathbf{Y}) = J(\mathbf{X}+a,\mathbf{Y}),\qquad
F(t,\mathbf{X},\mathbf{Y}) = J(t\cdot\mathbf{X},\mathbf{Y}).
\]

\subsection{B2}

Вводятся репрезентативные интервалы:
\[
\mathrm{mode}\,\mathbf{X},\qquad \mathrm{mode}\,\mathbf{Y}.
\]
Тогда
\[
F(a,\mathbf{X},\mathbf{Y}) = J(\mathrm{mode}\,\mathbf{X}+a,\mathrm{mode}\,\mathbf{Y}),\qquad
F(t,\mathbf{X},\mathbf{Y}) = J(t\cdot \mathrm{mode}\,\mathbf{X},\mathrm{mode}\,\mathbf{Y}).
\]

\subsection{B3}

Репрезентативные интервалы берутся как медианы Крейновича:
\[
\mathrm{med}_K\,\mathbf{X} = \bigl[\mathrm{med}(\underline{x_i}),\ \mathrm{med}(\overline{x_i})\bigr],\qquad
\mathrm{med}_K\,\mathbf{Y} = \bigl[\mathrm{med}(\underline{y_i}),\ \mathrm{med}(\overline{y_i})\bigr].
\]
Тогда
\[
F(a,\mathbf{X},\mathbf{Y}) = J(\mathrm{med}_K\,\mathbf{X}+a,\mathrm{med}_K\,\mathbf{Y}),\qquad
F(t,\mathbf{X},\mathbf{Y}) = J(t\cdot \mathrm{med}_K\,\mathbf{X},\mathrm{med}_K\,\mathbf{Y}).
\]

\subsection{B4}

Репрезентативные интервалы берутся как медианы Пролубникова:
\[
\mathrm{med}_P\,\mathbf{X},\qquad \mathrm{med}_P\,\mathbf{Y},
\]
где $\mathrm{med}_P$ выбирается по порядку центров интервалов (при равенстве центров — по ширине), и для чётного числа элементов берётся объединяющая оболочка двух центральных интервалов.
Тогда
\[
F(a,\mathbf{X},\mathbf{Y}) = J(\mathrm{med}_P\,\mathbf{X}+a,\mathrm{med}_P\,\mathbf{Y}),\qquad
F(t,\mathbf{X},\mathbf{Y}) = J(t\cdot \mathrm{med}_P\,\mathbf{X},\mathrm{med}_P\,\mathbf{Y}).
\]

\subsection{Вычисление интервальной моды}

Рассматриваются все границы интервалов $\{\underline{x_i},\overline{x_i}\}$ как события начала/конца покрытия.
По отсортированным событиям строится число активных интервалов $c(z)$.
Результат:
\[
\mathrm{mode}\,\mathbf{X} = \arg\max\limits_{z}\ c(z),
\]
при этом возвращается отрезок максимального покрытия (если он ненулевой), иначе точка максимального покрытия.

\section{Реализация}

Для поиска $\hat{s}$ параметр $s$ перебирался на равномерной сетке на заданном интервале значений.
Для каждой точки сетки вычислялось значение $F(s)$, после чего выбиралось
\[
\hat{s}=\arg\max F(s).
\]

\subsection{Интервальная оценка параметра по $\epsilon$}

Пусть $F_{\max}=\max F(s)$ на сетке. Тогда интервальная оценка строилась как
\[
S_\epsilon=\Bigl[\min\{s:\ F(s)\ge F_{\max}-\epsilon\},\ \max\{s:\ F(s)\ge F_{\max}-\epsilon\}\Bigr].
\]

\subsection{Особенности}

Для B2--B4 сначала вычислялись репрезентативные интервалы ($\mathrm{mode}$, $\mathrm{med}_K$, $\mathrm{med}_P$), после чего функционал считался только по ним, что уменьшает вычислительную сложность по сравнению с B1.


\textit{Код доступен по ссылке:} \url{https://github.com/shnikita-cr/2025-Interval-Analysis/tree/lab4/lab4}

\newpage

\section{Выводы}

\subsection{Результаты}

Вычисления проводились при $\epsilon = 0.001$.

$a \in [0.0, 1.0]$,

Поиск в пределах $t \in [-2.0, 0.0]$

\begin{tabular}{||c | c | c | c | c | c | c ||} 
	\hline
	Функционал  & $F(a)$ & $argmax\ F(a)$ & Интервал $a$ & $F(t)$ & $argmax\ F(t)$ &  Интервал $t$ \\ 
	\hline\hline
	$Ji(s,\bold{X}, \bold{Y})$ & $0.00210$ & 0.34000 & [0.33333, 0.35333] & $0.00208$ & -1.01333 & [-1.10667, -0.94667]\\
	\hline
	$Ji(s,mode\bold{X}, mode\bold{Y})$ & $0.99310$ & 0.34310 & [0.34310, 0.34310] & $0.99443$ & -1.00107 & [-1.00107, -1.00107]\\
	\hline
	$Ji(s,med_K\bold{X}, med_K\bold{Y})$ & $0.99212$ & 0.34353 & [0.34353, 0.34353] & $0.98554$ & -1.01467 & [-1.01468, -1.01467]\\
	\hline
	$Ji(s,med_P\bold{X}, med_P\bold{Y})$ & $0.99212$ & 0.34353 & [0.34353, 0.34353] & $0.98554$ & -1.01467 & [-1.01468, -1.01467]\\
	\hline
\end{tabular}


\subsection*{Вычисления для функционала $F_1$}

\includegraphics[scale=0.5]{images/b1/as.png}
\includegraphics[scale=0.5]{images/b1/am.png}

\subsection*{Вычисления для функционала $F_2$}

\includegraphics[scale=0.5]{images/b2/as.png}
\includegraphics[scale=0.5]{images/b2/am.png}

\subsection*{Вычисления для функционала $F_3$}

\includegraphics[scale=0.5]{images/b3/as.png}
\includegraphics[scale=0.5]{images/b3/am.png}

\subsection*{Вычисления для функционала $F_4$}

\includegraphics[scale=0.5]{images/b4/as.png}
\includegraphics[scale=0.5]{images/b4/am.png}


\subsection{Интерпретация результата}
\begin{itemize}
	\item $Ji(s,\bold{X}, \bold{Y})$ — оценка по всей выборке; максимум малый, интервалы параметров шире.
	\item $Ji(s,mode\bold{X}, mode\bold{Y})$ — оценка по модам; максимум близок к 1, интервалы параметров почти точечные.
	\item $Ji(s,med_K\bold{X}, med_K\bold{Y})$ — оценка по медианам Крейновича; близко к B2, но даёт немного иные $argmax$.
	\item $Ji(s,med_P\bold{X}, med_P\bold{Y})$ — оценка по медианам Пролубникова; в этих данных совпала с B3.
\end{itemize}

\end{document}


    