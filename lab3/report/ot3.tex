\documentclass[40pt,a4paper]{article}

\usepackage{graphicx, amsmath, amsfonts, listings}
\usepackage[T2A]{fontenc}
\usepackage[utf8]{inputenc}
\usepackage[russian,english]{babel}
\usepackage{geometry, tocloft}
\usepackage{hyperref}

\geometry{a4paper,left=15mm,top=15mm}

\graphicspath{{./images/}}

% Переименование оглавления
\addto\captionsenglish{\renewcommand{\contentsname}{Содержание}}

% Настройки межстрочных отступов в оглавлении
\setlength\cftparskip{1.2pt}
\setlength\cftbeforesecskip{1.3pt}
\setlength\cftaftertoctitleskip{2pt}

% Добавление точки после номера раздела
\makeatletter
\renewcommand{\@seccntformat}[1]{\csname the#1\endcsname.\quad}
\makeatother

% Добавление точек в оглавление
\let\savenumberline\numberline
\def\numberline#1{\savenumberline{#1.}}

\newcommand{\ran}{\mathrm{ran}}
\newcommand{\rad}{\mathrm{rad}}
\newcommand{\dist}{\mathrm{dist}}

\begin{document}

\begin{titlepage}
\centering

{\LARGE\bfseries Министерство образования и науки Российской Федерации}

\vspace{0.5cm}

{\LARGE\bfseries Санкт-Петербургский политехнический университет Петра Великого
}

\vspace{1.5cm}

{\Large Отчет по лабораторной работе №3 по дисциплине}
{\Large “Интервальный анализ”}

\vspace{1cm}

{\Large \textit{b-, A-, Ab-коррекция в линейной задаче о допусках}}

\vspace{16cm}

\begin{center}
\begin{tabular}{ r l c }
 Выполнил                       \\ 
 студент гр. 5030102/20201 & Н.А.Морщинин \\  
 преподаватель             & А.Н.Баженов
\end{tabular}
\end{center}
\vfill

\end{titlepage}

\thispagestyle{empty} 

\newpage

\tableofcontents

\newpage
\section{Введение}
\subsection{Постановка задачи}

Дан набор интервальных систем линейных уравнений: 
$\bold{A_i} x = \bold{b_i}$,
$x = (x_1, x_2); i=\overline{1,3}$

Для каждой системы $\bold{A_i} x = \bold{b_i}$ необходимо:

\begin{enumerate}
    \item Проверить непустоту допуского множества ИСЛАУ. В случае, если допусковое множество непусто:
    \begin{itemize}
    	\item Найти argmax Tol и образующие допускового функционала.
    	\item Построить график функционала $Tol(x)$, отметить точку максимума.
    	\item Построить графики допускового множества ИСЛАУ на плоскости, отметить точку максимума.
    \end{itemize}
    
    \item Если допусковое множество пусто, необходимо скорректировать систему каждым из описанных ниже способов: 
    \begin{itemize}
    	\item С помощью коррекции правой части ИСЛАУ - b-коррекция
    	\item С помощью коррекции левой части ИСЛАУ - А-коррекция
    	\item С помощью комбинации предыдущих методов с одновременным изменением правой и левой части ИСЛАУ - Ab-коррекция
    \end{itemize}
    \item Сравнить влияние разных видов коррекции на форму и положение допускового множества. Определить, какой вид коррекции является наименее искажением исходных данных при достижении разрешимости.
\end{enumerate}


\newpage

\subsection{Условия задачи}

$x \in X = [-10,10] \times [-10,10]$

\subsubsection{Система 1}

\[
A = \begin{pmatrix}
	[0.65, 1.25] & [0.70, 1.30] \\
	[0.75, 1.35] & [0.70, 1.30] \\
\end{pmatrix}, b = \begin{pmatrix}
	[2.75, 3.15] \\
	[2.85, 3.25] \\
\end{pmatrix}
\]

\subsubsection{Система 2}

\[
A = \begin{pmatrix}
	[0.65, 1.25] & [0.70  1.30] \\
	[0.75, 1.35] & [0.70, 1.30] \\
	[0.80, 1.40] & [0.70, 1.30] \\
\end{pmatrix}, b = \begin{pmatrix}
	[2.75, 3.15] \\
	[2.85, 3.25] \\
	[2.90, 3.30] \\
\end{pmatrix}
\]

\subsubsection{Система 3}

\[
A = \begin{pmatrix}
	[ 0.65, 1.25] & [0.70  1.30] \\
	[ 0.75, 1.35] & [0.70, 1.30] \\
	[ 0.80, 1.40] & [0.70, 1.30] \\
	[-0.30, 0.30] & [0.70, 1.30] \\
\end{pmatrix}, b = \begin{pmatrix}
	[2.75, 3.15] \\
	[2.85, 3.25] \\
	[2.90, 3.30] \\
	[1.80, 2.20] \\
\end{pmatrix}
\]

\subsection{Предварительный анализ}

Все системы имеют пустое допусковое множество без коррекции, это можно наблюдать на графиках ниже для каждой из систем. Изображен график $z_i = f(x_1, x_2) = (A_ix \subset b_i)$

\paragraph{Система 1 без коррекции}
\paragraph{}
\includegraphics[scale=0.25]{images/1/0_s.png}
\includegraphics[scale=0.25]{images/1/0_f.png}
\paragraph{Система 2 без коррекции}
\paragraph{}
\includegraphics[scale=0.25]{images/2/0_s.png}
\includegraphics[scale=0.25]{images/2/0_f.png}
\paragraph{Система 3 без коррекции}
\paragraph{}
\includegraphics[scale=0.25]{images/3/0_s.png}
\includegraphics[scale=0.25]{images/3/0_f.png}


\newpage

\section{Теория}

\subsection{b-коррекция}

Метод состоит в том, чтобы увеличить радиус компонент правой части, внеся добавку с коэффициентом C:
$ b = b + C * I$, где 

$ C = |maxTol(x,A,b)| + \epsilon$,

$I = ([-1,1], \dots,[-1,1])^T$,

$\epsilon$ задается пользователем для увеличения допускового множества. Здесь $\epsilon = 2$.

\textit{[Источник: Интервальный анализ. Основы теории и учебные примеры. А.Н.Баженов, 2020, стр. 50, пример 6.36]}

\subsection{A-коррекция}

Метод состоит в том, чтобы уменьшить радиус компонент матрицы, при этом оставив центры интервалов без изменений. В данном случае будем использовать единый коэффициент для всех строк матрицы.

$e \leftarrow E= [|\frac{maxTol(x,A,b)}{\sum{argmaxTol(x,A,b)}}|, min(rad(A))] $

Интервал $E$ получается из решения системы неравенств для A-коррекции.

В качестве $e$ возьмем правую границу интервала.

$I$ - матрица из интервалов вида $[-1,1]$ соответствующего $A$ размера.

$ A = A \ominus I*e $ 

\textit{[Источник: Интервальный анализ. Основы теории и учебные примеры. А.Н.Баженов, 2020, стр. 52, пример 6.38]}

\subsection{Ab-коррекция}

Ab-коррекция проводится последовательным изменением левой и правой частей по формулам, указанным в разделах выше. При этом поправка $\epsilon=0$.

\textit{[Источник: Комбинированный метод коррекции интервальных систем линейных алгебраических уравнений. А.Н.Баженов, А.Ю.Тельнова, 2022, стр. 974, выражения (18-20)]}

\section{Реализация}

Численная реализация была проведена по формулам, приведенным выше. 

\subsection{Алгоритм}

\begin{enumerate}
	\item Проверить пустоту допускового множества $(\exists x \in X: (\bold{A}x \subset \bold{b}))$
	\item Если не пусто:
	\begin{itemize}
		\item Вычислить значения допускового множества для $x \in X$
		\item Вычислить значения функционала $tol_{x \in X}(x,A,b)$, найти $max$, $argmax$ и образующие.
	\end{itemize}
	
	\item Если пусто:
	\begin{itemize}
		\item Провести b- коррекцию. Выполнить пункт 2. 
		\item Провести A- коррекцию. Выполнить пункт 2.
		\item Провести Ab- коррекцию. Выполнить пункт 2.
	\end{itemize}
\end{enumerate}

\textit{Код доступен по ссылке:} \url{https://github.com/shnikita-cr/2025-Interval-Analysis/tree/lab3/lab3}

\newpage

\section{Выводы}

\subsection{Результаты}

\subsubsection{Система 1}

\begin{tabular}{||c | c | c||} 
	\hline
	Тип коррекции  & $argmaxTol_{x \in X}(x,A,b)$ & $maxTol_{x \in X}(x,A,b)$\\ [0.5ex] 
	\hline\hline
	- &  (0.954 2.060) & -0.721  \\ 
	\hline
	b &  (0.954 2.060) & 0.556  \\ 
	\hline
	A &  (0.954 2.060) & 0.183  \\ 
	\hline
	Ab & (0.954 2.060) & 0.365  \\ 
	\hline
\end{tabular}

\subsubsection*{b коррекция}

\includegraphics[scale=0.4]{images/1/b_f.png}
\includegraphics[scale=0.4]{images/1/b_s.png}

\subsubsection*{A коррекция}

\includegraphics[scale=0.4]{images/1/a_f.png}
\includegraphics[scale=0.4]{images/1/a_s.png}

\subsubsection*{Ab коррекция}

\includegraphics[scale=0.4]{images/1/ab_f.png}
\includegraphics[scale=0.4]{images/1/ab_s.png}

\newpage

\subsubsection{Система 2}
\begin{tabular}{||c | c | c||} 
	\hline
	Тип коррекции  & $argmaxTol_{x \in X}(x,A,b)$ & $maxTol_{x \in X}(x,A,b)$\\ [0.5ex] 
	\hline\hline
	- &  (0.954 2.060) & -0.722  \\ 
	\hline
	b &  (0.954 2.060) & 0.556  \\ 
	\hline
	A &  (0.954 2.060) & 0.182  \\ 
	\hline
	Ab & (0.954 2.060) & 0.365  \\ 
	\hline
\end{tabular}

\subsubsection*{b коррекция}

\includegraphics[scale=0.4]{images/1/b_f.png}
\includegraphics[scale=0.4]{images/1/b_s.png}

\subsubsection*{A коррекция}

\includegraphics[scale=0.4]{images/1/a_f.png}
\includegraphics[scale=0.4]{images/1/a_s.png}

\subsubsection*{Ab коррекция}

\includegraphics[scale=0.4]{images/1/ab_f.png}
\includegraphics[scale=0.4]{images/1/ab_s.png}

\newpage

\subsubsection{Система 3}
\begin{tabular}{||c | c | c||} 
	\hline
	Тип коррекции  & $argmaxTol_{x \in X}(x,A,b)$ & $maxTol_{x \in X}(x,A,b)$\\ [0.5ex] 
	\hline\hline
	- &  (1.055 1.959) & -0.744  \\ 
	\hline
	b &  (1.055 1.959) & 0.510  \\ 
	\hline
	A &  (1.055 1.959) & 0.159  \\ 
	\hline
	Ab & (1.055 1.959) & 0.319  \\ 
	\hline
\end{tabular}

\subsubsection*{b коррекция}

\includegraphics[scale=0.4]{images/3/b_f.png}
\includegraphics[scale=0.4]{images/3/b_s.png}

\subsubsection*{A коррекция}

\includegraphics[scale=0.4]{images/3/a_f.png}
\includegraphics[scale=0.4]{images/3/a_s.png}

\subsubsection*{Ab коррекция}

\includegraphics[scale=0.4]{images/3/ab_f.png}
\includegraphics[scale=0.4]{images/3/ab_s.png}

\newpage

\subsection{Интерпретация результата}

А коррекция является минимальным искажением исходных данных задачи, все типы коррекций не смещают допусковое множество, а также точку максимума функционала $tol$. Заметим следующее:

\begin{itemize}
  \item \textbf{b-}коррекция расширяет допусковое множество в направлении главной диагонали $x_1=x_2$.
  \item \textbf{A-}коррекция расширяет допусковое множество в направлении побочной диагонали $x_1 = -x+2$, причем вытянутость сильно заметна и для допускового множества, и для графика $tol(x,A,b)$.
  \item \textbf{Ab-}коррекция также расширяет допусковое множество в направлении побочной диагонали $x_1 = -x+2$. Вытянутость сильно заметна только при условии систем 1 и 2. Из-за особенностей условия системы 3 (центрированный интервал в последней строке матрицы и вектора) допусковое множество получается довольно небольшим и не имеет заметной вытянутости, как в случае систем 1 и 2.
\end{itemize}

\end{document}


    