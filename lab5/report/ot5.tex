\documentclass[40pt,a4paper]{article}

\usepackage{graphicx, amsmath, amsfonts, listings}
\usepackage[T2A]{fontenc}
\usepackage[utf8]{inputenc}
\usepackage[russian,english]{babel}
\usepackage{geometry, tocloft}
\usepackage{hyperref}

\geometry{a4paper,left=15mm,top=15mm}

\graphicspath{{./images/}}

% Переименование оглавления
\addto\captionsenglish{\renewcommand{\contentsname}{Содержание}}

% Настройки межстрочных отступов в оглавлении
\setlength\cftparskip{1.2pt}
\setlength\cftbeforesecskip{1.3pt}
\setlength\cftaftertoctitleskip{2pt}

% Добавление точки после номера раздела
\makeatletter
\renewcommand{\@seccntformat}[1]{\csname the#1\endcsname.\quad}
\makeatother

% Добавление точек в оглавление
\let\savenumberline\numberline
\def\numberline#1{\savenumberline{#1.}}

\newcommand{\ran}{\mathrm{ran}}
\newcommand{\rad}{\mathrm{rad}}
\newcommand{\dist}{\mathrm{dist}}

\begin{document}

\begin{titlepage}
\centering

{\LARGE\bfseries Министерство образования и науки Российской Федерации}

\vspace{0.5cm}

{\LARGE\bfseries Санкт-Петербургский политехнический университет Петра Великого
}

\vspace{1.5cm}

{\Large Отчет по лабораторной работе №5 по дисциплине}
{\Large “Интервальный анализ”}

\vspace{1cm}

{\Large \textit{Решение интервальных систем нелинейных уравнений}}

\vspace{16cm}

\begin{center}
\begin{tabular}{ r l c }
 Выполнил                       \\ 
 студент гр. 5030102/20201 & Н.А.Морщинин \\  
 преподаватель             & А.Н.Баженов
\end{tabular}
\end{center}
\vfill

\end{titlepage}

\thispagestyle{empty} 

\newpage

\tableofcontents

\newpage
\section{Введение}
\subsection{Постановка задачи}

Задана система нелинейных уравнений, описывающая точку касания $f_1$ и $f_2$.

\[
f(X)=\begin{cases}
f_1(X) = X_1-X_2^2 = 0\\
f_2(X) = (X_1-x_c)^2 + (X_2-y_c)^2-1=0\\
\end{cases}
\]

где $x_c$ - параметр, принимающий значения из набора $\{0,0.5,1.0,1.2\}$, $y_c$=0.

\begin{enumerate}
	\item Для каждого значения параметра $x_c$ из заданного набора:
	\begin{enumerate}
		\item Выбрать начальное интервальное приближение $X^{(0)}=[X_1]\times[X_2]$, обеспечивающее старт итерационного процесса.
		\item Выполнить не менее трех итераций методом Кравчика для уточнения локализации решения.
		\item На каждом шаге фиксировать получающиеся интервальные векторы $X^{(k)}$
	\end{enumerate}
	\item Провести анализ результатов:
	\begin{enumerate}
		\item Для каждого $x_c$ построить графическую иллюстрацию, отображающую последовательные приближения (брусы $X^{(k)}$) и линии уровней функций $f_1(X)=0$ и $f_2(X)=0$.
		\item Сравнить скорость сходимости метода для разных значений параметра $x_c$. Объяснить наблюдаемые различия.
		\item Проанализировать влияние выбора начального приближения $X^{(0)}$ на возможность старта и сходимость итерационного процесса.
	\end{enumerate}
\end{enumerate}


% В ходе работы следует обратить внимание на возможное расширение интервалов на некоторых итерациях (как в приведённом примере для координаты x1) и связать это с геометрической интерпретацией задачи


\subsection{Условия задачи}

Начальное приближение: 

\begin{enumerate}
	\item $X_1 \times X_2 = [0.0,1.0]\times[0.0,1.0]$
	\item $X_1 \times X_2 = [0.5,1.5]\times[0.5,1.5]$
	\item $X_1 \times X_2 = [0.7,1.5]\times[0.7,1.5]$
	\item $X_1 \times X_2 = [0.4,0.6]\times[0.6,0.75]$
\end{enumerate}

\newpage

\section{Теория}

\paragraph{Метод Кравчика}

Рассматривается система нелинейных уравнений
\[
F_i(x)=0,\qquad F_i:\mathbb{R}^n\to\mathbb{R}^n.
\]
Пусть $X\subset\mathbb{R}^n$ --- интервальный вектор (брус), $x_0\in X$ --- опорная точка (обычно середина $x_0=\mathrm{mid}(X)$), а $J(X)$ --- интервальное расширение якобиана $F'(x)$ на $X$.
Выберем невырожденную матрицу $C\in\mathbb{R}^{n\times n}$, аппроксимирующую $(F'(x_0))^{-1}$
\[
C = \bigl(F'(x_0)\bigr)^{-1}.
\]

\paragraph{Оператор Кравчика.}
Оператор Кравчика $K(X)$ определяется формулой
\[
K(X) \;=\; x_0 - C\,F(x_0) \;+\; \bigl(I - C\,J(X)\bigr)\,(X - x_0),
\]
где $I$ --- единичная матрица, а произведения/суммы понимаются как интервальные (для матриц/векторов).

\paragraph{Итерационный процесс.}
Последовательность уточняющих брусьев строится пересечением
\[
X^{(k+1)} \;=\; X^{(k)} \cap K\!\left(X^{(k)}\right),\qquad k=0,1,2,\dots
\]
Если на некотором шаге $X^{(k+1)}=\emptyset$, то корней в исходном $X^{(k)}$ не содержится.

\paragraph{Критерий существования и единственности.}
Если выполнено строгое включение
\[
K(X)\subset \mathrm{int}(X),
\]
то в $X$ существует единственное решение $x^\ast$ уравнения $F(x)=0$.
(В частности, тогда итерации с пересечением дают сужающуюся локализацию решения.)


\textit{Код доступен по ссылке:} \url{https://github.com/shnikita-cr/2025-Interval-Analysis/tree/lab5/lab5}

\newpage

\section{Выводы}

\subsection{Результаты}

\subsection*{Вычисления для параметра $x_c = 0.0$}
\includegraphics[scale=0.5]{images/0.0/plot.png}
\includegraphics[scale=0.5]{images/0.0/x_k_iter.png}

\subsection*{Вычисления для параметра $x_c = 0.5$}
\includegraphics[scale=0.5]{images/0.5/plot.png}
\includegraphics[scale=0.5]{images/0.5/x_k_iter.png}

\subsection*{Вычисления для параметра $x_c = 1.0$}
\includegraphics[scale=0.5]{images/1.0/plot.png}
\includegraphics[scale=0.5]{images/1.0/x_k_iter.png}

\subsection*{Вычисления для параметра $x_c = 1.2$}
\includegraphics[scale=0.5]{images/1.2/plot.png}
\includegraphics[scale=0.5]{images/1.2/x_k_iter.png}


\subsection{Интерпретация результата}

\includegraphics[scale=0.5]{images/compare.png}
\begin{itemize}
	\item Лучшие сходимости при параметрах $x_c=0.0, x_c=0.5$, кривые явно пересекаются , наихудшая сходимость при параметра $x_c=1.2$, кривые неявно пересекаются, стремятся к касанию.
	\item По расположению точки пересечения и расположению кривых в случае первых трех параметров интервалы были выбраны довольно широкими, при них наблюдалась сходимость. Для четвертого случая для сходимости начальные интервалы пришлось существенно сузить, иначе итерационный процесс не начинался. 
	\item Не все интервалы возможны для выбора в качестве начальных: при нахождении двух корней в приближении или в случае, если в интервал $X_2$ попадал ноль, итерационный процесс не сходился или не начинался из-за особенностей метода (вычисление якобиана).
\end{itemize}

\end{document}


    