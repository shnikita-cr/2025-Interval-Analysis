\documentclass[40pt,a4paper]{article}

\usepackage{graphicx, amsmath, amsfonts, listings}
\usepackage[T2A]{fontenc}
\usepackage[utf8]{inputenc}
\usepackage[russian,english]{babel}
\usepackage{geometry, tocloft}

\geometry{a4paper,left=15mm,top=15mm}

\graphicspath{{./images/}}

% Переименование оглавления
\addto\captionsenglish{\renewcommand{\contentsname}{Содержание}}

% Настройки межстрочных отступов в оглавлении
\setlength\cftparskip{1.2pt}
\setlength\cftbeforesecskip{1.3pt}
\setlength\cftaftertoctitleskip{2pt}

% Добавление точки после номера раздела
\makeatletter
\renewcommand{\@seccntformat}[1]{\csname the#1\endcsname.\quad}
\makeatother

% Добавление точек в оглавление
\let\savenumberline\numberline
\def\numberline#1{\savenumberline{#1.}}

\newcommand{\ran}{\mathrm{ran}}
\newcommand{\rad}{\mathrm{rad}}
\newcommand{\dist}{\mathrm{dist}}

\begin{document}

\begin{titlepage}
\centering

{\LARGE\bfseries Министерство образования и науки Российской Федерации}

\vspace{0.5cm}

{\LARGE\bfseries Санкт-Петербургский политехнический университет Петра Великого
}

\vspace{1.5cm}

{\Large Отчет по лабораторной работе №2 по дисциплине}
{\Large “Интервальный анализ”}

\vspace{1cm}

{\Large \textit{Интервальные методы оценивания области значений функций на заданном отрезке.}}

\vspace{16cm}

\begin{center}
\begin{tabular}{ r l c }
 Выполнил                       \\ 
 студент гр. 5030102/20201 & Н.А.Морщинин \\  
 преподаватель             & А.Н.Баженов
\end{tabular}
\end{center}
\vfill

\end{titlepage}

\thispagestyle{empty} 

\newpage

\tableofcontents

\newpage
\section{Введение}
\subsection{Постановка задачи}
Даны две функции $f(x)$ и два интервала $X=[a.b]$ с соответствующими индексами. Необходимо исследовать и сравнить точности различных интервальных методов оценивания области значений функций на заданном отрезке.

\begin{enumerate}
\item Аналитически или численно найти область значений $\ran(f,X)$, построить график функции на заданном интервале.

\item Вычислить интервальные оценки области значений, используя:
  \begin{enumerate}
    \item Естественное интервальное расширение исходного выражения функции.
    \item Естественное интервальное расширение эквивалентного выражения функции, полученного с помощью схемы Горнера или иного алгебраического преобразования.
    \item Дифференциальную центрированную форму с центром в разных точках интервала.
    \item Наклонную центрированную форму с центром в разных точках интервала.
    \item Бицентрированную форму.
  \end{enumerate}

  \item Для каждой полученной интервальной оценки вычислить величину 
  $\dist(F(X),\ran(f,X))$ — расстояние по Хаусдорфу до точной области значений. 
  Проанализировать точность естественного интервального расширения:
  \begin{enumerate}
    \item Найти (аналитически или численно) константу Липшица $L$ для функции $f$ на интервале $X$. Обосновать свой выбор.
    \item Используя следствие из теоремы о непрерывности по Липшицу, получить теоретическую оценку погрешности:
    \[
      \rad\!\bigl(F(X)\bigr) \;\le\; L\,\rad(X).
    \]
    \item Сравнить реальную погрешность (полуширину полученного интервала $\rad(F(X))$) с теоретической оценкой из пункта (б). Сделать выводы.
  \end{enumerate}

  \item Сравнить и проанализировать результаты, объяснив наблюдаемую точность или неточность каждого метода.
\end{enumerate}

\newpage

\subsection{Условия задачи}

Функция 1: $f_1=x^3-3x^2+2$, $X=[0,3]$
Функция 2: $f_2=x^5-2x^3+sin(x)$, $X=[0,1.5]$

\paragraph{График функции $f_1$ на отрезке $X_1$}
\paragraph{}

\includegraphics[scale=0.6]{images/f1.png}

\paragraph{График функции $f_2$ на отрезке $X_2$}
\paragraph{}

\includegraphics[scale=0.6]{images/f2.png}

\newpage

\section{Теория}

\subsection{Естественное интервальное расширение исходного выражения}
\textit{Описание:} исходное выражение $f$ с интервальными операциями. \quad
\[
F_{1}(X) := f_i(X).
\]

\subsection{Естественное интервальное расширение эквивалентого выражения}
\textit{Описание:} схема Горнера для полинома $f(x)=\sum_{k=0}^{n} a_k x^k$. \quad
\[
F_{2}(X) := (((a_n X + a_{n-1})X + \cdots + a_1)X + a_0).
\]

Далее в эквивалентное выражение подставляется исходный интервал $X$.

\subsection{Дифференциальная центрированная форма}
\textit{Описание:} среднезначная форма с центром $m\in X$, $F'(X)$ — интервальная производная. \quad
\[
F_{3}(X;m) := f(m) + F'(X)\,(X - m), \quad m \in \{a,\frac{a+b}{2},b\}
\]

\subsection{Наклонная центрированная форма}

\paragraph{Определим наклон}
\[
S(x,m)=\frac{f(x)-f(m)}{\,x-m\,}\quad (x\in X,\ x\neq m).
\]

\textit{Описание:} двусторонняя наклонная форма, $X_L=[a,m],\ X_R=[m,b]$, $S_L := F'(X_L),\ S_R:= F'(X_R)$. \quad
\[
F_{4}(X;m) := \bigl(f(m) + S_L\,(X_L - m)\bigr)\ \cup \ \bigl(f(m) + S_R\,(X_R - m)\bigr),
\]
Разбиение на два интервала используется для того, чтобы обойти деление на интервал, содержащий ноль.

\subsection{Бицентрированная форма}
\textit{Описание:} теорема Баумана; $m=\operatorname{mid}X$, $r=\operatorname{rad}X$, $F'(X)$ — интервальная производная. \quad
\[
p=\operatorname{cut}\!\bigl(\operatorname{mid}F'(X),[-1,1]\bigr),\quad
c_{\downarrow}=m-pr,\quad c_{\uparrow}=m+pr,
\]
\[
\mathrm{fmv}(X;c)=f(c)+F'(X)(X-c),\qquad
F_{5}(X)=\mathrm{fmv}(X;c_{\downarrow})\ \cap\ \mathrm{fmv}(X;c_{\uparrow}).
\]




\newpage

\section{Реализация}

Численная реализация была проведена по формулам, приведенным выше. 

\subsection{Алгоритм}
Для каждого пункта из постановки задачи вычислялась оценка, ширина полученного интервала, погрешность, а также проверялось неравенство из следствия о непрерывности по Липшицу. Также проводилось сравнение с исходным (аналитическим) интервалом значений функций.

\section{Выводы}

\subsection{Результаты для $f_1(x)$ на $X=[0,3]$}
Точная область значений: $\ran(f_1,X)=\bigl[-1.99997,\,2.00000\bigr]$, \quad $L=\sup_{x\in X}|f_1'(x)|=9$, \quad $\rad(X)=1.5$, \; $L\cdot\rad(X)=13.5$.

\begin{center}
\renewcommand{\arraystretch}{1.2}
\begin{tabular}{l|c|c|c|c}
\hline
\textbf{Метод} & \textbf{Оценка $F(X)$} & \textbf{Погрешность $\rad(F(X))$} & \textbf{$\rad\le L\rad$?} & \textbf{$\dist(F(X),\ran)$} \\
\hline
B1 & $\left[-25.00000,\ 29.00000\right]$ & $27.00000$ & нет & $27.00000$ \\
B2 & $\left[-25.00000,\ \;\,2.00000\right]$ & $13.50000$ & да  & $23.00003$ \\
B3 & $\left[-14.87500,\ 12.12500\right]$ & $13.50000$ & да  & $12.87503$ \\
B4 & $\left[\ -3.29200,\ \ \ 6.13100\right]$ & $4.71150$  & да  & $4.13100$  \\
B5 & $\left[\ -6.01562,\ \ \ 4.95312\right]$ & $5.48438$  & да  & $4.01566$  \\
\hline
\end{tabular}
\end{center}

\vspace{0.5em}
\noindent\ Наименьшую ширину дала B4, а наименьшее расстояние до $\ran(f_1,X)$ — B5.

\subsection{Результаты для $f_2(x)$ на $X=[0,1.5]$}
Точная область значений: $\ran(f_2,X)=\bigl[-0.17163,\,1.84124\bigr]$, \quad $L=26.31250$, \quad $\rad(X)=0.75$, \quad $L\,\rad(X)=19.73438$.

\begin{center}
\renewcommand{\arraystretch}{1.2}
\begin{tabular}{l|c|c|c|c}
\hline
\textbf{Метод} & \textbf{Оценка $F(X)$} & \textbf{Погрешность $\rad(F(X))$} & \textbf{$\rad\le L\rad$?} & \textbf{$\dist(F(X),\ran)$} \\
\hline
B1 & $\left[-6.75000,\ 8.59124\right]$  & $7.67062$  & да & $6.75000$ \\
B2 & $\left[-6.75000,\ 1.84124\right]$  & $4.29562$  & да & $6.57837$ \\
B3 & $\left[-19.65918,\ 19.80957\right]$&  $\begin{matrix} 19.73438\ (\text{исходный}) \\ 10.07195\ (\text{схема\ Горнера})  \end{matrix}    $ & да & $19.48756$ \\
B4 & $\left[-7.60295,\ 8.46624\right]$  & $8.03459$  & да & $7.43133$ \\
B5 & $\left[-13.07856,\ 13.18180\right]$& $13.13018$ & да & $12.90693$ \\
\hline
\end{tabular}
\end{center}

\vspace{0.5em}
\noindent Наименьшая ширина — у B2; минимальная дистанция до $\ran(f_2,X)$ — также у B2.


\subsection{Интерпретация результата}

\begin{itemize}
  \item \textbf{B1:} самое широкое оценивание; базовый ориентир, иногда нарушает Липшица.
  \item \textbf{B2:} схема Горнера улучшает оценку; лучше всего для $f_2$.
  \item \textbf{B3:} при широком $F'(X)$ вырождается в $L\,\rad(X)$; большие значения на немонотонных участках.
  \item \textbf{B4:} двусторонняя наклонная форма обычно уже B3; дала минимальную ширину для $f_1$.
  \item \textbf{B5:} пересечение двух MV; хорошо при узком $F'(X)$ (лучшая дистанция для $f_1$), средне для $f_2$.
  \item Полученные оценки можно улучшать с помощью подбора центров, изменения представления функции и других модификаций. Для примера был проведен эксперимент с \textbf{B3}: при использовании схемы Горнера для модификации представления функции оценка улучшается почти в 2 раза.
\end{itemize}

\end{document}


    